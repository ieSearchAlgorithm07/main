\subsection{Leve1.2: 評価方法(目的関数の設計指針や方法)について}
\subsubsection{課題説明}
Amazonにおける書籍検索時に「ファンタジー作品で泣ける作品」を探し出すため
のアイテム集合xと目的関数f(x)について検討した。

\subsubsection{アイテム集合xについて}
私達は、映画や書籍、ゲームからファンタジー作品について検討した。まずファンタジー作品とは、モンスターや魔法など現実に存在しないものが登場することや、想像上の物語が展開されていることが挙げられると考えた。

\subsubsection{目的関数について}
レビューから、「泣く」や「感動する」といった単語を抽出し、さらに、現実に存在しないものが登場することが目的関数になると考えた。これらに点数をつけ、どれほどファンタジー作品に近いかを点数から評価することができる

%(補足:PDF図を挿入する例)

%\begin{figure}[h]
% \begin{center}
%  \includegraphics[width=8.0cm]{figs/system-image.pdf}
%  \caption{入出力と内部モデルのイメージ図}
% \end{center}
%\end{figure}


\subsection{Level 3.1: $y = x^2 + (y^2)/10$ について}
この関数は1つの谷を持つ関数であるが,最急降下法では上手く最小値を見つける事ができない.
\subsubsection{原因}
最急降下法は勾配ベクトルの逆を進み,山を下るような感覚で最小値を発見しようとするものである.しかし勾配ベクトルには,実際の最小値への方向とは誤差が含まれている.それを刻み値$\alpha$で修正するが,方向にずれが生じているため結果的にジグザグに動くことになる.これは$x$方向と$y$方向の勾配に差があるためであり,結果的には勾配が大きい方に重みが生じていることになる.
\subsubsection{改善方法}
この欠点を改善するためには「共役勾配法」という,前回の進行方向を加味して次の方向を決めるようなアルゴリズムである.これは単純に計算量が増えるものの比較的効率よく最小値を探索することができる.



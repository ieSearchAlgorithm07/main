\subsection{Level2.1,2.2 共通部分 観察意図と観察方法}
最急降下法の学習レートであるalphaを任意で初期値,刻み幅,最大値を決め連続関数における探索プログラムを実行し,シード値1000〜10000におけるstep数とその平均を出力してくれるスクリプトalpha.shを使用し,alphaを0.01〜0.99と刻み幅0.01で変更しながら
step数の平均を観察する.
これらの方法により効率的に最適解(最小値となる解)を求められる,alphaの値の発見およびalphaの変化による
探索への影響を考察することを目的とする.

\subsection{Level2.1: $y=x^2$ について}
\subsubsection{プログラムソース(変更部分)}

	\begin{lstlisting}[caption=探索プログラム,label=ラベル]
	#include <stdio.h>
	#include <stdlib.h>
	#include <math.h>

	-------------省略----------------

	double f(double x, double y) {
  	double z;

  	/** 以下の式を編集して完成させよ(1) **/
  	z = x*x;

  	return( z );
	}

	/* f(x,y)/dx
 	*    z=f(x,y) の微分値(偏微分値)を求め,返す.
 	*/
	double pd_x(double x, double y) {
  	double z_dx;

  	/** 以下の式を編集して完成させよ(2-1) **/
  	z_dx = 2*x;

  	return( z_dx );
	}

	/* f(x,y)/dy
 	*    z=f(x,y) の微分値(偏微分値)を求め,返す.
 	*/
	double pd_y(double x, double y) {
  	double z_dy;

  	/** 以下の式を編集して完成させよ(2-2) **/
  	z_dy = 0;

  	return( z_dy );
	}

	-------------省略----------------

    return 0;
	}
	\end{lstlisting}



\subsubsection{実行結果}

  \begin{lstlisting}[caption=シェルスクリプトalpha.sh実行結果,label=ラベル]
	alpha = 0.01
	FINISH 3 step 592 x and y were not updated.
	sim 1: seed=1000 -> step=591
	FINISH 3 step 712 x and y were not updated.
	sim 2: seed=2000 -> step=711
	FINISH 3 step 646 x and y were not updated.
	sim 3: seed=3000 -> step=645
	FINISH 3 step 702 x and y were not updated.
	sim 4: seed=4000 -> step=701
	FINISH 3 step 671 x and y were not updated.
	sim	5: seed=5000 -> step=670
	FINISH 3 step 539 x and y were not updated.
	sim 6: seed=609000 -> step=538
	FINISH 3 step 688 x and y were not updated.
	sim 7: seed=7000 -> step=687
	FINISH 3 step 674 x and y were not updated.
	sim 8: seed=8000 -> step=673
	FINISH 3 step 701 x and y were not updated.
	sim 9: seed=9000 -> step=700
	FINISH 3 step 650 x and y were not updated.
	sim 10: seed=10000 -> step=649
	average step = 656.50

	-------------省略----------------
	
	alpha = .50
	
	FINISH 3 step 2 x and y were not updated.
	sim 1: seed=1000 -> step=1
	FINISH 3 step 2 x and y were not updated.
	sim 2: seed=2000 -> step=1
	FINISH 3 step 2 x and y were not updated.
	sim 3: seed=3000 -> step=1
	FINISH 3 step 2 x and y were not updated.
	sim 4: seed=4000 -> step=1
	FINISH 3 step 2 x and y were not updated.
	sim 5: seed=5000 -> step=1
	FINISH 3 step 2 x and y were not updated.
	sim 6: seed=609000 -> step=1
	FINISH 3 step 2 x and y were not updated.
	sim 7: seed=7000 -> step=1
	FINISH 3 step 2 x and y were not updated.
	sim 8: seed=8000 -> step=1
	FINISH 3 step 2 x and y were not updated.
	sim 9: seed=9000 -> step=1
	FINISH 3 step 2 x and y were not updated.
	sim 10: seed=10000 -> step=1
	average step = 1.00
	
	-------------省略----------------
	
	alpha = .99
	FINISH 3 step 819 x and y were not updated.
	sim 1: seed=1000 -> step=818
	FINISH 3 step 939 x and y were not updated.
	sim 2: seed=2000 -> step=938
	FINISH 3 step 874 x and y were not updated.
	sim 3: seed=3000 -> step=873
	FINISH 3 step 929 x and y were not updated.
	sim 4: seed=4000 -> step=928
	FINISH 3 step 899 x and y were not updated.
	sim 5: seed=5000 -> step=898
	FINISH 3 step 766 x and y were not updated.
	sim 6: seed=609000 -> step=765
	FINISH 3 step 916 x and y were not updated.
	sim 7: seed=7000 -> step=915
	FINISH 3 step 901 x and y were not updated.
	sim 8: seed=8000 -> step=900
	FINISH 3 step 928 x and y were not updated.
	sim 9: seed=9000 -> step=927
	FINISH 3 step 877 x and y were not updated.
	sim 10: seed=10000 -> step=876
	average step = 883.80
	finish
  \end{lstlisting}

  \begin{lstlisting}[caption=探索プログラム実行結果,label=ラベル]
  %./steepest_decent2_1 1000 
  step 0 x 0.7557628633 y 2.1064439007 f(x,y) 0.1775056 5.711775e-01
  step 1 x 0.0000000000 y 2.1064439007 f(x,y) 0.0000000000 0.000000e+00
  FINISH 3 step 2 x and y were not updated.
  \end{lstlisting}
  ※シード値:1000 学習レートα:0.5(最も平均step数が少なかったαの探索結果)



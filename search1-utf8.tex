%\documentclass[a4paper,10pt]{jarticle}
%\documentclass[]{jarticle}
\documentclass[10pt]{jarticle}

%\usepackage{graphicx}
\usepackage[dvipdfmx]{graphicx}
\usepackage{eclbkbox} %breakbox用

\usepackage{listings,jlisting}
%listingsの設定
\renewcommand{\lstlistingname}{ソースコード}
\lstset{language=c,
  breaklines = true 
  basicstyle=\ttfamily\scriptsize,
  commentstyle=\textit,
  classoffset=1,
  showstringspaces=false
  keywordstyle=\bfseries,
  frame=tRBl,
  framesep=5pt,
  showstringspaces=false,
  numbers=left,
  stepnumber=1,
  numberstyle=\tiny,
  tabsize=2
}

%本文領域を広め(空白箇所マージン領域を小さめ)に設定
\setlength{\textwidth}{179mm}
\setlength{\textheight}{251mm}
\setlength{\topmargin}{-2cm}
\setlength{\oddsidemargin}{-1cm}
\setlength{\evensidemargin}{-1cm}

\begin{document}

\title{情報工学実験IIレポート(探索アルゴリズム1)}
\author{金曜日&グループ7} %
\date{\today}

\maketitle
\section*{グループメンバ}
\begin{itemize}
 \item 135712D 山城義弘: 担当Level1.1, 1.2
 \item 135706K 喜友名 朝稔: 担当Level2.1, 2.2
 \item 135719B 澤崎夏希: 担当Level2.3, 3.1
 \item 135740K 大城朝貴: 担当Level4.1, 4.2
\end{itemize}

\section*{提出したレポート一式について}
レポート一式は
``\verb|shell:/net/home/teacher/tnal/2013-search1-mon/group0/|''
にアップロードした。
提出したファイルのディレクトリ構成は以下の通りである。

\vspace{+0.5cm}
自分たちでやりやすいようにLevel毎に整理しても構わない)
\begin{breakbox}
\begin{verbatim}
./scr/      # 作成したプログラム一式
./report/   # レポート関係ファイル.図ファイルを含む.
\end{verbatim}
\end{breakbox}

\newpage
\section{Leve l1: 探索とは}
以下は『人工知能研究\cite{level1}』を参考に記述した.
\subsection{Level1.1: コンピュータと人間の違いを述べよ}
\subsubsection{課題説明}
コンピュータが人間より得意とするモノ、その反対に人間より不得手のモノ、両者について2つ以上の視点(立場や観点など)を示し、考察する。

\subsubsection{考察}
\begin{itemize}
 \item 視点1: 計算、処理速度\\
数学などの単純計算において、人間は算盤などの道具を使っても、計算速度はコンピュータが勝る。
 \item 視点2: 学習、推測\\
コンピュータはある物事について完全に正しいデータを与えられないと、正しいルールを得ることができない。人間には勘や感情があるため、論理的に説明できないこともこなすことができる。
 \item 視点3: 作業効率\\
ベルトコンベアなどで行われる単純作業は、コンピュータ(ロボット)でも人力でも行われているが、人間には集中力などの観点から、長時間の作業は難しい。よって、生産量などの結果はコンピュータが勝る。
\end{itemize}

\subsection{Leve1.2: 評価方法(目的関数の設計指針や方法)について}
\subsubsection{課題説明}
Amazonにおける書籍検索時に「ファンタジー作品で泣ける作品」を探し出すため
のアイテム集合xと目的関数f(x)について検討した。

\subsubsection{アイテム集合xについて}
私達は、映画や書籍、ゲームからファンタジー作品について検討した。まずファンタジー作品とは、モンスターや魔法など現実に存在しないものが登場することや、想像上の物語が展開されていることが挙げられると考えた。

\subsubsection{目的関数について}
レビューから、「泣く」や「感動する」といった単語を抽出し、さらに、現実に存在しないものが登場することが目的関数になると考えた。これらに点数をつけ、どれほどファンタジー作品に近いかを点数から評価することができる

%(補足:PDF図を挿入する例)

%\begin{figure}[h]
% \begin{center}
%  \includegraphics[width=8.0cm]{figs/system-image.pdf}
%  \caption{入出力と内部モデルのイメージ図}
% \end{center}
%\end{figure}



\newpage

\section{Level 2: 最急降下法による最適化}
以下は『最急降下法\cite{level2}』を参考に記述した.
\subsection{Level2.1,2.2 共通部分 観察意図と観察方法}
最急降下法の学習レートであるalphaを任意で初期値,刻み幅,最大値を決め連続関数における探索プログラムを実行し,シード値1000〜10000におけるstep数とその平均を出力してくれるスクリプトalpha.shを使用し,alphaを0.01〜0.99と刻み幅0.01で変更しながら
step数の平均を観察する.
これらの方法により効率的に最適解(最小値となる解)を求められる,alphaの値の発見およびalphaの変化による
探索への影響を考察することを目的とする.

\subsection{Level2.1: $y=x^2$ について}
\subsubsection{プログラムソース(変更部分)}

	\begin{lstlisting}[caption=探索プログラム,label=ラベル]
	#include <stdio.h>
	#include <stdlib.h>
	#include <math.h>

	-------------省略----------------

	double f(double x, double y) {
  	double z;

  	/** 以下の式を編集して完成させよ(1) **/
  	z = x*x;

  	return( z );
	}

	/* f(x,y)/dx
 	*    z=f(x,y) の微分値(偏微分値)を求め,返す.
 	*/
	double pd_x(double x, double y) {
  	double z_dx;

  	/** 以下の式を編集して完成させよ(2-1) **/
  	z_dx = 2*x;

  	return( z_dx );
	}

	/* f(x,y)/dy
 	*    z=f(x,y) の微分値(偏微分値)を求め,返す.
 	*/
	double pd_y(double x, double y) {
  	double z_dy;

  	/** 以下の式を編集して完成させよ(2-2) **/
  	z_dy = 0;

  	return( z_dy );
	}

	-------------省略----------------

    return 0;
	}
	\end{lstlisting}



\subsubsection{実行結果}

  \begin{lstlisting}[caption=シェルスクリプトalpha.sh実行結果,label=ラベル]
	alpha = 0.01
	FINISH 3 step 592 x and y were not updated.
	sim 1: seed=1000 -> step=591
	FINISH 3 step 712 x and y were not updated.
	sim 2: seed=2000 -> step=711
	FINISH 3 step 646 x and y were not updated.
	sim 3: seed=3000 -> step=645
	FINISH 3 step 702 x and y were not updated.
	sim 4: seed=4000 -> step=701
	FINISH 3 step 671 x and y were not updated.
	sim	5: seed=5000 -> step=670
	FINISH 3 step 539 x and y were not updated.
	sim 6: seed=609000 -> step=538
	FINISH 3 step 688 x and y were not updated.
	sim 7: seed=7000 -> step=687
	FINISH 3 step 674 x and y were not updated.
	sim 8: seed=8000 -> step=673
	FINISH 3 step 701 x and y were not updated.
	sim 9: seed=9000 -> step=700
	FINISH 3 step 650 x and y were not updated.
	sim 10: seed=10000 -> step=649
	average step = 656.50

	-------------省略----------------
	
	alpha = .50
	
	FINISH 3 step 2 x and y were not updated.
	sim 1: seed=1000 -> step=1
	FINISH 3 step 2 x and y were not updated.
	sim 2: seed=2000 -> step=1
	FINISH 3 step 2 x and y were not updated.
	sim 3: seed=3000 -> step=1
	FINISH 3 step 2 x and y were not updated.
	sim 4: seed=4000 -> step=1
	FINISH 3 step 2 x and y were not updated.
	sim 5: seed=5000 -> step=1
	FINISH 3 step 2 x and y were not updated.
	sim 6: seed=609000 -> step=1
	FINISH 3 step 2 x and y were not updated.
	sim 7: seed=7000 -> step=1
	FINISH 3 step 2 x and y were not updated.
	sim 8: seed=8000 -> step=1
	FINISH 3 step 2 x and y were not updated.
	sim 9: seed=9000 -> step=1
	FINISH 3 step 2 x and y were not updated.
	sim 10: seed=10000 -> step=1
	average step = 1.00
	
	-------------省略----------------
	
	alpha = .99
	FINISH 3 step 819 x and y were not updated.
	sim 1: seed=1000 -> step=818
	FINISH 3 step 939 x and y were not updated.
	sim 2: seed=2000 -> step=938
	FINISH 3 step 874 x and y were not updated.
	sim 3: seed=3000 -> step=873
	FINISH 3 step 929 x and y were not updated.
	sim 4: seed=4000 -> step=928
	FINISH 3 step 899 x and y were not updated.
	sim 5: seed=5000 -> step=898
	FINISH 3 step 766 x and y were not updated.
	sim 6: seed=609000 -> step=765
	FINISH 3 step 916 x and y were not updated.
	sim 7: seed=7000 -> step=915
	FINISH 3 step 901 x and y were not updated.
	sim 8: seed=8000 -> step=900
	FINISH 3 step 928 x and y were not updated.
	sim 9: seed=9000 -> step=927
	FINISH 3 step 877 x and y were not updated.
	sim 10: seed=10000 -> step=876
	average step = 883.80
	finish
  \end{lstlisting}

  \begin{lstlisting}[caption=探索プログラム実行結果,label=ラベル]
  %./steepest_decent2_1 1000 
  step 0 x 0.7557628633 y 2.1064439007 f(x,y) 0.1775056 5.711775e-01
  step 1 x 0.0000000000 y 2.1064439007 f(x,y) 0.0000000000 0.000000e+00
  FINISH 3 step 2 x and y were not updated.
  \end{lstlisting}
  ※シード値:1000 学習レートα:0.5(最も平均step数が少なかったαの探索結果)



\subsection{Level2.2: $z=x^2 + y^2$ について}
\subsubsection{プログラムソース(変更部分)}

	\begin{lstlisting}[caption=探索プログラム,label=ラベル]
	#include <stdio.h>
	#include <stdlib.h>
	#include <math.h>

	-------------省略----------------

	double f(double x, double y) {
  	double z;

  	/** 以下の式を編集して完成させよ(1) **/
  	z = x*x+y*y;

  	return( z );
	}

	/* f(x,y)/dx
 	*    z=f(x,y) の微分値(偏微分値)を求め,返す.
 	*/
	double pd_x(double x, double y) {
  	double z_dx;

  	/** 以下の式を編集して完成させよ(2-1) **/
  	z_dx = 2*x;

  	return( z_dx );
	}

	/* f(x,y)/dy
 	*    z=f(x,y) の微分値(偏微分値)を求め,返す.
 	*/
	double pd_y(double x, double y) {
  	double z_dy;

  	/** 以下の式を編集して完成させよ(2-2) **/
  	z_dy = 2*y;

  	return( z_dy );
	}

	-------------省略----------------

    return 0;
	}
	\end{lstlisting}

\subsubsection{観察意図と観察方法}

\subsubsection{実行結果}
	\begin{lstlisting}[caption=シェルスクリプトalpha.sh実行結果,label=ラベル]
	alpha = 0.01
	FINISH 3 step 643 x and y were not updated.
	sim 1: seed=1000 -> step=642
	FINISH 3 step 712 x and y were not updated.
	sim 2: seed=2000 -> step=711
	FINISH 3 step 697 x and y were not updated.
	sim 3: seed=3000 -> step=696
	FINISH 3 step 702 x and y were not updated.
	sim 4: seed=4000 -> step=701
	FINISH 3 step 717 x and y were not updated.
	sim 5: seed=5000 -> step=716
	FINISH 3 step 690 x and y were not updated.
	sim 6: seed=6000 -> step=689
	FINISH 3 step 688 x and y were not updated.
	sim 7: seed=7000 -> step=687
	FINISH 3 step 701 x and y were not updated.
	sim 8: seed=8000 -> step=700
	FINISH 3 step 701 x and y were not updated.
	sim 9: seed=9000 -> step=700
	FINISH 3 step 714 x and y were not updated.
	sim 10: seed=10000 -> step=713
	average step = 695.50

	-------------省略----------------
	
	alpha = .50
	FINISH 3 step 2 x and y were not updated.
	sim 1: seed=1000 -> step=1
	FINISH 3 step 2 x and y were not updated.
	sim 2: seed=2000 -> step=1
	FINISH 3 step 2 x and y were not updated.
	sim 3: seed=3000 -> step=1
	FINISH 3 step 2 x and y were not updated.
	sim 4: seed=4000 -> step=1
	FINISH 3 step 2 x and y were not updated.
	sim 5: seed=5000 -> step=1
	FINISH 3 step 2 x and y were not updated.
	sim 6: seed=6000 -> step=1
	FINISH 3 step 2 x and y were not updated.
	sim 7: seed=7000 -> step=1
	FINISH 3 step 2 x and y were not updated.
	sim 8: seed=8000 -> step=1
	FINISH 3 step 2 x and y were not updated.
	sim 9: seed=9000 -> step=1
	FINISH 3 step 2 x and y were not updated.
	sim 10: seed=10000 -> step=1
	average step = 1.00

	-------------省略----------------
	
	alpha = .99
	FINISH 3 step 870 x and y were not updated.
	sim 1: seed=1000 -> step=869
	FINISH 3 step 939 x and y were not updated.
	sim 2: seed=2000 -> step=938
	FINISH 3 step 924 x and y were not updated.
	sim 3: seed=3000 -> step=923
	FINISH 3 step 929 x and y were not updated.
	sim 4: seed=4000 -> step=928
	FINISH 3 step 944 x and y were not updated.
	sim 5: seed=5000 -> step=943
	FINISH 3 step 917 x and y were not updated.
	sim 6: seed=6000 -> step=916
	FINISH 3 step 916 x and y were not updated.
	sim 7: seed=7000 -> step=915
	FINISH 3 step 928 x and y were not updated.
	sim 8: seed=8000 -> step=927
	FINISH 3 step 928 x and y were not updated.
	sim 9: seed=9000 -> step=927
	FINISH 3 step 942 x and y were not updated.
	sim 10: seed=10000 -> step=941
	average step = 922.70
	\end{lstlisting}

  \begin{lstlisting}[caption=探索プログラム実行結果,label=ラベル]
  %./steepest_decent2_2 1000 
  step 0 x 0.7557628633 y 2.1064439007 f(x,y) 5.0082834122 5.008283e+00
  step 1 x 0.0000000000 y 0.0000000000 f(x,y) 0.0000000000 0.000000e+00
  FINISH 3 step 2 x and y were not updated.
  \end{lstlisting}
  ※シード値:1000 学習レートα:0.5(最も平均step数が少なかったαの探索結果)

\subsection{Level2.1,2.2 共通部分 考察}

作成したシェルスクリプトalpha.shを実行した結果,当初の予想と同じように学習レートアルファが小さければ詳細に探索でき,アルファが大きくなれば大雑把だが効率よく最適解付近にたどり着くことができることがわかった.今回は,alphaを初期値,刻み幅,最大値をきめ動かしたがしらみつぶしに探索したが,何カ所かの異なるalphaにおいての探索を行うことにより,ある程度効率の良い探索ができる学習レートを予想することができる.

\subsection{Level2.3: $y=-x*sin(x)$ について}
この関数はいくつかの峰と谷を持つ複雑な関数であり,谷を見つけてもそこが最低値かどうかは判断出来ない.そこで以下のようのソースコードを修正した.
\subsubsection{プログラムソース(変更部分)}
\begin{lstlisting}[caption=変更部分,label=fix01]
--------------...
83     n=4;
84     d=X_RANGE/n;
85
86     for (j=0; j < n;j++){
87       x = X_MIN+d*j + (X_RANGE/n) * (double)rand()/RAND_MAX;
...------------
\end{lstlisting}
\subsubsection{観察意図と観察方法}
 これは,最急降下法で観察する部分を分割することで,様々な場所の最小値を発見しようとしている.変更したのは初期値だけであるが,最急降下法は初期値の発生位置によって結果が大きく左右されるため十分であると判断した.観察は当初出力結果を観察していたが,データの推移が判断しにくいためgnuplotでのグラフとの比較を見て判断した.グラフの作成にはシェルスクリプト(スクリプト2.3参照)を用いた.
\subsubsection{実行結果}

\begin{figure}[htb]
 \begin{center}
 \includegraphics[width=10.0cm]{./level2/level2.3.eps}
  \caption{修正されたプログラムによる出力結果}
  \label{level2.3}
 \end{center}
\end{figure}
\subsubsection{考察}
上の結果が示すように分割した位置で最急降下法を実行することにより,それぞれの谷の位置を発見することが出来た.今回は谷の数を考慮したn(分割数)=4でプログラムを実行したが,現実的にはnもパラメータとして定義し,都合の良い値を推測べきである.



\newpage

\section{Level 3: 最急降下法が苦手とする状況}
\subsection{Level 3.1: $y = x^2 + (y^2)/10$ について}
この関数は1つの谷を持つ関数であるが,最急降下法では上手く最小値を見つける事ができない.
\subsubsection{原因}
最急降下法は勾配ベクトルの逆を進み,山を下るような感覚で最小値を発見しようとするものである.しかし勾配ベクトルには,実際の最小値への方向とは誤差が含まれている.それを刻み値$\alpha$で修正するが,方向にずれが生じているため結果的にジグザグに動くことになる.これは$x$方向と$y$方向の勾配に差があるためであり,結果的には勾配が大きい方に重みが生じていることになる.
\subsubsection{改善方法}
この欠点を改善するためには「共役勾配法」という,前回の進行方向を加味して次の方向を決めるようなアルゴリズムである.これは単純に計算量が増えるものの比較的効率よく最小値を探索することができる.




\newpage

\section{Level 4: モデル推定時における目的関数の設計}
以下は『最適化問題\cite{level401}』参考に記述した.
Housing Data Set\cite{housingdata}$B$rNc$K!"(B
$B%b%G%k$NE,@Z$5$r?^$k$?$a$NL\E*4X?t$KIU$$$F@_7W$7$?!#(B

\subsection{$BL\E*4X?t$K$D$$$F(B}
$B%b%G%k$N4X?t$r(B
f(x)=10*x-40\\
$B$H$7!"(B\\
f(x)$B$+$iI}(B25$BN%$l$?$H$3$m$K0J2<$N$h$&$J4X?t$r2>Dj$9$k$H!"(B\\
g(x)=$<$f(x+5)+5\\
g(x)=$>$f(x-5)-5\\
$B$NHO0OFb$K$"$k@VE@$N8D?t$,B?$$Dx%b%G%k$OE,@Z$G$"$k!#$3$l$O:GBg2=$rMxMQ$7$?L\E*4X?t$G$"$k!#(B\\

$B$^$?!":G>.2=$rMxMQ$9$k>l9g$O!"0J2<$N$h$&$K(B\\
g(x)=$>$f(x+5)+5\\
g(x)=$<$f(x-5)-5\\
$B$H$9$k$H!"HO0OFb$K$"$k@VE@$N8D?t$,>/$J$$Dx%b%G%k$OE,@Z$G$"$k$H8@$($k!#(B\\
$B:#2s$O:GBg2=$rMxMQ$7$?L\E*4X?t$r9M$($k!#(B\\

\subsection{$B@_7WM}M3$K$D$$$F(B}
$B<+J,$?$A$N%0%k!<%W$G$O%b%G%k$H3F@VE@$^$G$N5wN%$r2C;;$7$?7k2L$GI>2A$9$kL\E*4X?t$H%b%G%k$+$i$"$k0lDj$N5wN%Fb$K$"$k@VE@$N?t$GI>2A$9$kL\E*4X?t$NFs$D$r9M$($^$7$?$,!"8e<T$NJ}$,I>2A$,4J7i$GJ,$+$j$d$9$$$H$NM}M3$+$i!"8e<T$rA*Br$7$^$7$?!#(B\\
$B$J$<!"$3$N$h$&$JL\E*4X?t$,8uJd$K5s$,$C$?$N$+$H$$$&$H!":GE,$J%b%G%k$H$O%0%i%UA4BN$N@VE@$NJ,I[$N798~$rDO$s$@$b$N$G!"%b%G%k$N0lDj5wN%Fb$KA4$F$N@VE@$,B8:_$7$F$k$b$N$G$"$k$H9M$($^$9!#(B\\
$B$3$N%b%G%k$NI>2AJ}K!$G$O!"D>@~$N%b%G%k$@$1$G$J$/!"MM!9$J7A$N%b%G%k$KBP$9$kL\E*4X?t$rMF0W$K@_Dj$9$k$3$H$,=PMh$^$9!#(B\\
\input{level4/level4.2}


\vspace{+1.0cm}
\begin{thebibliography}{99}
\bibitem{info2-search1}
情報工学実験2: 探索アルゴリズムその1(當間)\\
\verb|http://www.eva.ie.u-ryukyu.ac.jp/~tnal/2013/info2/search1/|
\bibitem{level1}
人工知能研究\\
\verb|http://www.ai-gakkai.or.jp/whatsai/AIresearch.html|
\bibitem{level2}
最急降下法\\
\verb|http://dsl4.eee.u-ryukyu.ac.jp/DOCS/nlp/node4.html|
\bibitem{level401}
最適化問題\\
\verb|http://www.wikiwand.com/ja/%E6%9C%80%E9%81%A9%E5%8C%96%E5%95%8F%E9%A1%8C|
\bibitem{level402}
シェルスクリプト リファレンス\\
\verb|http://shellscript.sunone.me/|

\end{thebibliography}

\end{document}

\subsection{Level2.2: $z=x^2 + y^2$ について}
\subsubsection{プログラムソース(変更部分)}

	\begin{lstlisting}[caption=探索プログラム,label=ラベル]
	#include <stdio.h>
	#include <stdlib.h>
	#include <math.h>

	-------------省略----------------

	double f(double x, double y) {
  	double z;

  	/** 以下の式を編集して完成させよ(1) **/
  	z = x*x+y*y;

  	return( z );
	}

	/* f(x,y)/dx
 	*    z=f(x,y) の微分値(偏微分値)を求め,返す.
 	*/
	double pd_x(double x, double y) {
  	double z_dx;

  	/** 以下の式を編集して完成させよ(2-1) **/
  	z_dx = 2*x;

  	return( z_dx );
	}

	/* f(x,y)/dy
 	*    z=f(x,y) の微分値(偏微分値)を求め,返す.
 	*/
	double pd_y(double x, double y) {
  	double z_dy;

  	/** 以下の式を編集して完成させよ(2-2) **/
  	z_dy = 2*y;

  	return( z_dy );
	}

	-------------省略----------------

    return 0;
	}
	\end{lstlisting}

\subsubsection{観察意図と観察方法}

\subsubsection{実行結果}
	\begin{lstlisting}[caption=シェルスクリプトalpha.sh実行結果,label=ラベル]
	alpha = 0.01
	FINISH 3 step 643 x and y were not updated.
	sim 1: seed=1000 -> step=642
	FINISH 3 step 712 x and y were not updated.
	sim 2: seed=2000 -> step=711
	FINISH 3 step 697 x and y were not updated.
	sim 3: seed=3000 -> step=696
	FINISH 3 step 702 x and y were not updated.
	sim 4: seed=4000 -> step=701
	FINISH 3 step 717 x and y were not updated.
	sim 5: seed=5000 -> step=716
	FINISH 3 step 690 x and y were not updated.
	sim 6: seed=6000 -> step=689
	FINISH 3 step 688 x and y were not updated.
	sim 7: seed=7000 -> step=687
	FINISH 3 step 701 x and y were not updated.
	sim 8: seed=8000 -> step=700
	FINISH 3 step 701 x and y were not updated.
	sim 9: seed=9000 -> step=700
	FINISH 3 step 714 x and y were not updated.
	sim 10: seed=10000 -> step=713
	average step = 695.50

	-------------省略----------------
	
	alpha = .50
	FINISH 3 step 2 x and y were not updated.
	sim 1: seed=1000 -> step=1
	FINISH 3 step 2 x and y were not updated.
	sim 2: seed=2000 -> step=1
	FINISH 3 step 2 x and y were not updated.
	sim 3: seed=3000 -> step=1
	FINISH 3 step 2 x and y were not updated.
	sim 4: seed=4000 -> step=1
	FINISH 3 step 2 x and y were not updated.
	sim 5: seed=5000 -> step=1
	FINISH 3 step 2 x and y were not updated.
	sim 6: seed=6000 -> step=1
	FINISH 3 step 2 x and y were not updated.
	sim 7: seed=7000 -> step=1
	FINISH 3 step 2 x and y were not updated.
	sim 8: seed=8000 -> step=1
	FINISH 3 step 2 x and y were not updated.
	sim 9: seed=9000 -> step=1
	FINISH 3 step 2 x and y were not updated.
	sim 10: seed=10000 -> step=1
	average step = 1.00

	-------------省略----------------
	
	alpha = .99
	FINISH 3 step 870 x and y were not updated.
	sim 1: seed=1000 -> step=869
	FINISH 3 step 939 x and y were not updated.
	sim 2: seed=2000 -> step=938
	FINISH 3 step 924 x and y were not updated.
	sim 3: seed=3000 -> step=923
	FINISH 3 step 929 x and y were not updated.
	sim 4: seed=4000 -> step=928
	FINISH 3 step 944 x and y were not updated.
	sim 5: seed=5000 -> step=943
	FINISH 3 step 917 x and y were not updated.
	sim 6: seed=6000 -> step=916
	FINISH 3 step 916 x and y were not updated.
	sim 7: seed=7000 -> step=915
	FINISH 3 step 928 x and y were not updated.
	sim 8: seed=8000 -> step=927
	FINISH 3 step 928 x and y were not updated.
	sim 9: seed=9000 -> step=927
	FINISH 3 step 942 x and y were not updated.
	sim 10: seed=10000 -> step=941
	average step = 922.70
	\end{lstlisting}

  \begin{lstlisting}[caption=探索プログラム実行結果,label=ラベル]
  %./steepest_decent2_2 1000 
  step 0 x 0.7557628633 y 2.1064439007 f(x,y) 5.0082834122 5.008283e+00
  step 1 x 0.0000000000 y 0.0000000000 f(x,y) 0.0000000000 0.000000e+00
  FINISH 3 step 2 x and y were not updated.
  \end{lstlisting}
  ※シード値:1000 学習レートα:0.5(最も平均step数が少なかったαの探索結果)

\subsection{Level2.1,2.2 共通部分 考察}

作成したシェルスクリプトalpha.shを実行した結果,当初の予想と同じように学習レートアルファが小さければ詳細に探索でき,アルファが大きくなれば大雑把だが効率よく最適解付近にたどり着くことができることがわかった.今回は,alphaを初期値,刻み幅,最大値をきめ動かしたがしらみつぶしに探索したが,何カ所かの異なるalphaにおいての探索を行うことにより,ある程度効率の良い探索ができる学習レートを予想することができる.
